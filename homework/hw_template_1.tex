\documentclass[12pt]{article}
\usepackage[margin=1.3in]{geometry}
\usepackage{xeCJK}
\setmonofont{SourceCodePro-Regular}
\setCJKmainfont[BoldFont=FandolSong-Bold,ItalicFont=FandolKai-Regular]{FandolSong-Regular}
\setCJKsansfont[BoldFont=FandolHei-Bold]{FandolHei-Regular}
\setCJKmonofont{FandolFang-Regular}
\usepackage{graphicx}
\usepackage{caption}
\usepackage{subcaption}
\usepackage{amsmath,amscd,amsbsy,amssymb,latexsym,url,bm,amsthm}
\usepackage{tikz}
\usetikzlibrary{automata,arrows}
\usepackage{enumerate}
\usepackage{float}
\usepackage{pifont}
\usepackage{color}
\usepackage{changepage}
\usepackage{diagbox}
\usepackage{placeins}
\usepackage{dcolumn}
\usepackage{listings}
\usepackage{fontspec}

\newtheorem{problem}{Problem}[]
\newtheorem*{solution}{Solution}
\makeatletter \renewenvironment{proof}[1][Proof] {\par\pushQED{\qed}\normalfont\topsep6\p@\@plus6\p@\relax\trivlist\item[\hskip\labelsep\bfseries#1\@addpunct{.}]\ignorespaces}{\popQED\endtrivlist\@endpefalse} \makeatother
\makeatletter \renewenvironment{solution}[1][Solution] {\par\pushQED{\qed}\normalfont\topsep6\p@\@plus6\p@\relax\trivlist\item[\hskip\labelsep\bfseries#1\@addpunct{.}]\ignorespaces}{\popQED\endtrivlist\@endpefalse} \makeatother

\begin{document}
  \title{\textbf{<TITLE>}}
    \author{<AUTHOR NAME>\\<STUDENT ID>}
    \maketitle

    \begin{problem}[x.x]
      <insert text here>            %  \textcolor{ _COLOR_ }{ _TYPE_TEXT_HERE_ }
      \lstset{                    % Add code.
        language=C, 
        basicstyle=\fontspec{Consolas}}
        \begin{lstlisting}
          <insert code here> 
        \end{lstlisting}

      \begin{table}[ht]
      \newcommand{\tabincell}[2]{\begin{tabular}{@{}#1@{}}#2\end{tabular}}
        \centering
        \begin{tabular}{|l|c|r|}
          \hline  
          \tabincell{c}{A1\\A2\\A3} &
          \tabincell{c}{B1\\B2} &
          C \\
          \hline 
          A&B&C\\
          \hline 
        \end{tabular}
      \end{table}
      
      \[A+B=C\]                   % Add simple formula.
        
      \[                          % Add complex formula.
        \begin{split}
        \text{A}=&~B_{C}\\
        &+D^{E}\times \sqrt{F} \\
        &+\frac{G}{H} \\
        \approx &I
        \end{split}
      \]
      
      \begin{tabular}{cD{.}{.}{3}}% Another way.
         &A\\
        +&B\\
        \hline
         &C\\
      \end{tabular}
    \end{problem}
    
    \begin{solution}
      \begin{enumerate}[a.]
        \item <insert text here> 
        \item <insert text here> s
      \end{enumerate}
    \end{solution}
\end{document}
