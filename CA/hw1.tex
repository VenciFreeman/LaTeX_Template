\documentclass[12pt]{article}
\usepackage[margin=1.3in]{geometry}
\usepackage{xeCJK}
\setmonofont{SourceCodePro-Regular}
\setCJKmainfont[BoldFont=FandolSong-Bold,ItalicFont=FandolKai-Regular]{FandolSong-Regular}
\setCJKsansfont[BoldFont=FandolHei-Bold]{FandolHei-Regular}
\setCJKmonofont{FandolFang-Regular}
\usepackage{graphicx}
\usepackage{caption}
\usepackage{subcaption}
\usepackage{amsmath,amscd,amsbsy,amssymb,latexsym,url,bm,amsthm}
\usepackage{tikz}
\usetikzlibrary{automata,arrows}
\usepackage{enumerate}
\usepackage{float}
\usepackage{pifont}
\usepackage{color}
\usepackage{changepage}
\usepackage{diagbox}
\usepackage{placeins}

\newtheorem{problem}{Problem}[]
\newtheorem*{solution}{Solution}
\makeatletter \renewenvironment{proof}[1][Proof] {\par\pushQED{\qed}\normalfont\topsep6\p@\@plus6\p@\relax\trivlist\item[\hskip\labelsep\bfseries#1\@addpunct{.}]\ignorespaces}{\popQED\endtrivlist\@endpefalse} \makeatother
\makeatletter \renewenvironment{solution}[1][Solution] {\par\pushQED{\qed}\normalfont\topsep6\p@\@plus6\p@\relax\trivlist\item[\hskip\labelsep\bfseries#1\@addpunct{.}]\ignorespaces}{\popQED\endtrivlist\@endpefalse} \makeatother

\begin{document}
  \title{\textbf{Homework 1}}
    \author{杨文曦\\
      517030910221}
    \maketitle

    \begin{problem}[1.2]
      The eight great ideas in computer architecture are similar to ideas from other fields. Match the eight ideas from computer architecture, “Design for Moore’s Law,” “Use Abstraction to Simplify Design,” “Make the Common Case Fast,” “Performance via Parallelism,” “Performance via Pipelining,” “Performance via Prediction,” “Hierarchy of Memories,” and “Dependability via Redundancy” to the following ideas from other fields:
      \begin{enumerate}[a.]
        \item Assembly lines in automobile manufacturing
        \item Suspension bridge cables
        \item Aircraft and marine navigation systems that incorporate wind information
        \item Express elevators in buildings
        \item Library reserve desk
        \item Increasing the gate area on a CMOS transistor to decrease its switching time
        \item Adding electromagnetic aircraft catapults (which are electrically powered as opposed to current steam-powered models), allowed by the increased power generation offered by the new reactor  technology
        \item Building self-driving cars whose control systems partially rely on existing sensor systems already installed into the base vehicle, such as lane departure systems and smart cruise control systems
      \end{enumerate}
    \end{problem}

    \begin{solution}
      The matching result is as follows:
      \begin{enumerate}[a.]
        \item Performance via Pipelining
        \item Performance via Parallelism \textcolor{red}{Dependability via Redundancy}
        \item Performance via Prediction
        \item Make the Common Case Fast
        \item Hierarchy of Memories
        \item Design for Moore’s Law \textcolor{red}{Performance via Parallelism}
        \item Dependability via Redundancy \textcolor{red}{Design for Moore’s Law}
        \item Use Abstraction to Simplify Design
      \end{enumerate}
    \end{solution}

    \begin{problem}[1.5]
      Consider three different processors P1, P2, and P3 executing the same instruction set. P1 has a 3.0 GHz clock rate and a CPI of 1.5. P2 has a 2.5 GHz clock rate and a CPI of 1.0. P3 has a 4.0 GHz clock rate and has a CPI of 2.2.
      \begin{enumerate}[a.]
        \item Which processor has the highest performance expressed in instructions per second?
        \item If the processors each execute a program in 10 seconds, find the number of cycles and the number of instructions.
        \item We are trying to reduce the execution time by 30\%, but this leads to an increase of 20\% in the CPI. What clock rate should we have to get this time reduction?
      \end{enumerate}
    \end{problem}

    \begin{solution}
      \begin{enumerate}[a.]
        \item For P1, instructions per second is     \textcolor{cyan}{Caution! Unit}
        \[\frac{3.0\text{ GHz}}{1.5\text{ CPI}} = 2.0\times 10^9\]
        And for P2, it's
        \[\frac{2.5\text{ GHz}}{1.0\text{ CPI}} = 2.5\times 10^9\]
        And for P3, it's
        \[\frac{4.0\text{ GHz}}{2.2\text{ CPI}} \approx 1.8\times 10^9\]
        So P2 has the highest performance.
        \item We know that
        \[\text{Number of CPU Clock Cycles} = \text{Time}\times \text{Clock Rate}\]
        So for P1, the number of cycles
        \[10\text{ s}\times 3.0\text{ GHz} = 3.0\times 10^{10}\]
        And the number of instructions is
        \[\frac{\text{Clock cycles}}{\text{CPI}}=2.0\times 10^{10}\]
        Similarly, we know \par 
        for P2, the number of cycles
        \[10\text{ s}\times 2.5\text{ GHz} = 2.5\times 10^{10}\]
        And the number of instructions is
        \[\frac{\text{Clock cycles}}{\text{CPI}}=2.5\times 10^{10}\]
        And for P3 the number of cycles is
        \[10\text{ s}\times 4\text{ GHz} = 4.0\times 10^{10}\]
        And the number of instructions is
        \[\frac{\text{Clock cycles}}{\text{CPI}}=1.8\times 10^{10}\]
        \item Cause we know
        \[\text{CPU Execution Time} = \frac{\text{Number of Instructions}\times \text{CPI}}{\text{Clock rate}}\]
        So 
        \[120\%\ \text{CPU Execution Time} = \frac{\text{Number of Instructions}\times70\%\ \text{CPI}}{\text{Clock Rate}}\]
        \[\text{Clock Rate} = \frac{120\%}{70\%}\ \text{Old Clock Rate}\]
        So clock rate should be about 1.7 times the original value.
      \end{enumerate}
    \end{solution}

    \begin{problem}[1.8]
      The Pentium 4 Prescott processor, released in 2004, had a clock rate of 3.6 GHz and voltage of 1.25 V. Assume that, on average, it consumed 10 W of static power and 90 W of dynamic power.\\
      The Core i5 Ivy Bridge, released in 2012, has a clock rate of 3.4 GHz and voltage of 0.9 V. Assume that, on average, it consumed 30 W of static power and 40 W of dynamic power.
      \begin{enumerate}[a.]
        \item For each processor find the average capacitive loads.
        \item Find the percentage of the total dissipated power comprised by static power and the ratio of static power to dynamic power for each technology.
        \item If the total dissipated power is to be reduced by 10\%, how much should the voltage be reduced to maintain the same leakage current? Note: power is defined as the product of voltage and current.
      \end{enumerate}
    \end{problem}

    \begin{solution}
      \begin{enumerate}[a.]
        \item The average capacitive for Pentium 4 Prescott is 
        \[C=\frac{P}{V^2f}=\frac{90\text{ W}}{1.25\text{ V}^2\times 3.6\text{ GHz}} = 1.6\times 10^-8\text{ C/V}=16\text{ nF}\]
        Similarly, the average capacitive load for the Core i5 Ivy Bridge is
        \[C=\frac{P}{V^2f}=\frac{40\text{ W}}{0.9\text{ V}^2\times 3.4\text{ GHz}}\approx 1.45\times 10^-8\text{ C/V}=14.5\text{ nF}\]
        \item For Pentium Prescott, the pencentage of the total dissipated power comrised by static power is
        \[\frac{10\text{ W}}{10\text{ W} + 90\text{ W}}=10\%\]
        And the ratio of static to dynamic power is
        \[\frac{10\text{ W}}{90\text{ W}}\approx 11.1\%\]
        Similarly, for the Core i5 Ivy Bridge, the pencentage of the total dissipated power comrised by static power is
        \[\frac{30\text{ W}}{30\text{ W} + 40\text{ W}}\approx 42.86\%\]
        And the ratio of static to dynamic power is
        \[\frac{30\text{ W}}{40\text{ W}} = 75\%\]
        \item Cause
        \[P_{total} = P_{static} + P_{dynamic} = VI_D+CV^2f=90\%\ \times (10\text{ W} + 90\text{ W}) = 90 \text{ W}\]
        And we know 
        \[I_D = \frac{P_{static}}{V_{old}} = \frac{10\text{ W}}{1.25\text{ V}} = 8 \text{ A}\]
        So we can list the equation
        \[(16\text{ nF}\times 3.6\text{ GHz})V^2 + 8\text{ A}\times V- 90\text{ W} = 0\]
        And
        \[V=\frac{-8\text{ A}\pm \sqrt{8\text{ A}^2-4\times (16\text{ nF}\times 3.6\text{ GHz})(-90\text{ W})}}{2(16\text{ nF}\times 3.6\text{ GHz})}\]
        Abandon the negative solution, we get
        \[V\approx 1.18\text{ V}\]
        Similarly, for the Core i5 Ivy Bridge,
        \[V\approx 0.84\text{ V}\]
      \end{enumerate}
    \end{solution}

  \begin{problem}[1.8]
    Assume a 15 cm diameter wafer has a cost of 12, contains 84 dies, and has 0.020 $\text{defects/cm}^2$. Assume a 20 cm diameter wafer has a cost of 15, contains 100 dies, and has 0.031 $\text{defects/cm}^2$.
    \begin{enumerate}[a.]
      \item Find the yield for both wafers.
      \item Find the cost per die for both wafers.
      \item If the number of dies per wafer is increased by 10\% and the defects per area unit increases by 15\%, find the die area and yield.
      \item Assume a fabrication process improves the yield from 0.92 to 0.95. Find the defects per area unit for each version of the technology given a die area of 200 $\text{mm}^2$.
    \end{enumerate}
  \end{problem}

  \begin{solution}
    \begin{enumerate}[a.]
      \item For 15 cm diameter wafer,
      \[\text{area per die} = \frac{\pi (\frac{15\text{ cm}^2}{2})^2}{84}\approx 2.1\text{ cm}^2\]
      So \textcolor{cyan}{based on empirical observations of yields}
      \[\text{yield} = \frac{1}{(1+0.020\text{ defects/cm}^2 \times \frac{2.1\text{ cm}^2}{2})^2}\approx 95.9\%\]
      Similarly, for 20 cm diameter wafer,
      \[\text{area per die} = \frac{\pi (\frac{20\text{ cm}^2}{2})^2}{100}\approx \pi \text{ cm}^2\]
      So
      \[\text{yield} = \frac{1}{(1+0.031\text{ defects/cm}^2 \times \frac{\pi\text{ cm}^2}{2})^2}\approx 90.9\%\]
      \item For 15 cm diameter wafer,
      \[\text{cost per die} = \frac{12}{84\times 95.9\%}\approx 0.149\]
      And for 20 cm diameter wafer,
      \[\text{cost per die} = \frac{15}{100\times 90.9\%}\approx 0.165\]
      \item For 15 cm diameter wafer,  \textcolor{cyan}{an approximation}
      \[\text{area per die} = \frac{2.1\text{ cm}^2}{110\%}\approx 1.91\text{ cm}^2\]
      And
      \[\text{yield} = \frac{1}{(1+115\%\times 0.020\text{ defects/cm}^2 \times \frac{1.91\text{ cm}^2}{2})^2}\approx 95.7\%\]
      Similarly, for 20 cm diameter wafer,
      \[\text{area per die} = \frac{\pi\text{ cm}^2}{110\%}\approx 2.86\text{ cm}^2\]
      And 
      \[\text{yield} = \frac{1}{(1+115\%\times 0.031\text{ defects/cm}^2 \times \frac{2.86\text{ cm}^2}{2})^2}\approx 90.5\%\]
      \item \[\text{yield} = \frac{1}{(1+\text{Defect Rate}\times frac{200\text{ mm}^2}{2})^2}\]
      So 
      \[\text{yield} = \frac{1}{(1+\text{Defect Rate})}\Longrightarrow\text{Defect Rate}= \frac{1}{\sqrt{\text{yield}}}-1\]
      So when yield is 0.92,
      \[\text{Defect Rate}= \frac{1}{\sqrt{\text{0.92}}}-1\approx 0.043\text{ defects/cm}^2\]
      And when yield is 0.95,
      \[\text{Defect Rate}= \frac{1}{\sqrt{\text{0.95}}}-1\approx 0.026\text{ defects/cm}^2\]
    \end{enumerate}
  \end{solution}
\end{document}
